\RequirePackage[l2tabu, orthodox]{nag}
\documentclass[a4paper, twocolumn]{article}
\usepackage[utf8]{inputenc}
\usepackage[T1]{fontenc}
\usepackage[pdftex, hidelinks,
            pdftitle={Behaviour Tree Evolution by Genetic Programming},
            pdfauthor={Martin Estgren and Erik S. V. Jansson},
            pdfsubject={Artificial Intelligence - Genetic Programming},
            pdfkeywords={artificial intelligence, genetic programming,
                         behaviour trees, a-star, shooter}]{hyperref}

\usepackage{bm}
\usepackage{caption}
\usepackage{listings}
\usepackage{booktabs}
\usepackage{mathtools}
\usepackage{algorithmic}
\usepackage{graphicx}
\usepackage{courier}
\usepackage{amsmath}
\usepackage{amssymb}
\usepackage{algorithm}
\usepackage[capitalize, noabbrev]{cleveref}
\usepackage[activate={true, nocompatibility}, final,
            tracking=true, kerning=true, spacing=true,
            factor=1100, stretch=10, shrink=10]{microtype}

\DeclareCaptionFormat{modifiedlst}{\rule{\textwidth}{0.85pt}\\[-2.9pt]#1#2#3}
\captionsetup[lstlisting]{format =  modifiedlst,
labelfont=bf,singlelinecheck=off,labelsep=space}
\lstset{basicstyle=\footnotesize\ttfamily,
        breakatwhitespace = false,
        breaklines = true,
        keepspaces = true,
        language = Java,
        showspaces = false,
        showstringspaces = false,
        frame = tb,
        numbers = left,
        numbersep = 5pt,
        xleftmargin = 16pt,
        framexleftmargin = 16pt,
        belowskip = \bigskipamount,
        aboveskip = \bigskipamount,
        escapeinside={<@}{@>}}

\title{\textbf{Behaviour Tree Evolution by Genetic Programming}\\
       \Large{\emph{-- Learning Novel Bot Behaviours in a 2D Top-Down Arena Shooter --}}}
\author{{\textbf{Martin Estgren}} \;\;\;\;\;\;\;\;\;\, {\href{mailto:mares480@student.liu.se}
                                                       {\texttt{<mares480@student.liu.se>}}} \\
        {\textbf{Erik S. V. Jansson}} \;\;\;\;         {\href{mailto:erija578@student.liu.se}
                                                       {\texttt{<erija578@student.liu.se>}}} \\~\\
        {Linköping University, Sweden}\vspace{-2.0ex}}

\begin{document}
    \maketitle
    \section*{Abstract}

    Behaviour trees are a popular model for representing the decision-making and plan execution process for NPCs in video games. These are built by hand, and require expertise to craft. They don't adapt well to other environments, and often require a custom BT. \footnote{Repository: \url{https://github.com/sci10n/Quake2D}}

    \tableofcontents \newpage

    \newpage % Next column...
    \nocite{*} % Include all.
    \bibliographystyle{abbrv}
    \bibliography{report}
    \clearpage

    \section{Introduction} \label{sec:introduction}



        \subsection{The Motivation} \label{sec:the_motivation}



        \subsection{Proposed Approach} \label{sec:proposed_approach}



    \section{Background Theory} \label{sec:background_theory}



        \subsection{Behaviour Trees} \label{sec:behaviour_trees}



        \subsection{Path Finding with A*} \label{sec:path_finding}



        \subsection{Genetic Programming} \label{sec:genetic_programming}



    \section{Implementation Details} \label{sec:implementation_details}



        \subsection{Game Architecture} \label{sec:game_architecture}



        \subsection{Behaviour Trees} \label{sec:behaviour_trees_implementation}



        \subsection{Path Finding with A*} \label{sec:path_finding_implementation}



        \subsection{Genetic Programming} \label{sec:genetic_programming_implementation}



    \section{Results and Screenshots} \label{sec:results_and_screenshots}



        \subsection{Generated Behaviours} \label{sec:generated_behaviours}



        \subsection{Behaviour Fitness} \label{sec:behaviour_fitness}



        \subsection{Survival Rate} \label{sec:survival_rate}



    \section{Discussion and Outlook} \label{sec:discussion_and_outlook}



\end{document}
